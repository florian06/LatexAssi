\documentclass[10pt,a4paper]{article}
\usepackage{color}
\usepackage{url}
\usepackage{enumitem}
\usepackage{graphicx}
\usepackage{listings}
\usepackage{verbatim}
\usepackage{array}
\usepackage{caption}
\usepackage[utf8]{inputenc}
\usepackage{pdfpages}

\author{Patrick de Niet, Joseph Hill and Florian Ecard}
    
\title{\LaTeX}
    
\begin{document}
    
\begin{figure}[t]
\centering
\includegraphics[width=7.6cm]{uva.jpg}\\
\textbf{{MSc. System and Network Engineering}}
\end{figure}
\maketitle


\newpage
\section{Introduction}
\paragraph{}


\newpage
\tableofcontents


\newpage
\section{Homework1}

\subsection{What is a wiki?}

\paragraph{}"A federated wiki is a collection of federated wiki servers. They have the ability to share pages among one another. This kind of wiki aims to prove that if it had been implemented since the wiki creation, it would have been better.\\
This project was started in 2012 in Portland at the Indie camp \cite{floref2}."\cite{floref1}

\subsection{What is a federated wiki?}

\paragraph{}"Ward Cunningham is the wiki's inventor, he made it in 1992.
A wiki is a web application where the users can read, add, edit or delete something that was created by anyone else.\\
As Wikipedia confirms it, a wiki is best used as an encyclopedia \cite{floref3}."\cite{floref1}

\subsection{What is the difference between them?}
\paragraph{}"The biggest difference between these two is the way the information is displayed to the user. A federated wiki is much more adapted to deeper analysis and ivestigation.\\
A normal wiki (e.g. Wikipédia), the whole content of a wiki is displayed, and its included link will redirect you to a different page. On the other hand, a federated wiki displays a wiki article within a third of a page. Therefore, when clicking on a link, another third of the screen would be diplayed. Showing different links in the same screen. There are no found limits (tested only up to 5 links) of wiki pages in a federated wiki. You will just have to scroll horizontally to find the desired wiki.\\
Last but not least, a federated wiki can host content onto different servers, whereas a simple wiki would host the whole in a single server \cite{floref1}\cite{floref3}."\cite{floref1}

\paragraph{Sources}
\begin{itemize}
\item \cite{floref1}
\item \cite{floref2} 
\item \cite{floref3}
\end{itemize}




\newpage
\section{Homework2}
\paragraph{}
% Patrick's part



\newpage
\section{Homework3}
\paragraph{}
% Joseph's part





\newpage
\listoffigures

\newpage
\bibliography{bib}{}
\bibliographystyle{plain}


\end{document}

